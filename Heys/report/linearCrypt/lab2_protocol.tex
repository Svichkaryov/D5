\documentclass[12pt,a4paper]{article}
\usepackage[utf8]{inputenc}
\usepackage[english,russian,ukrainian]{babel}
\usepackage{indentfirst}
\usepackage{misccorr}
\usepackage{graphicx}
\usepackage{amsmath}
\usepackage{euscript}  
\usepackage{slashbox}
\renewcommand{\baselinestretch}{1}

\usepackage{xcolor}
\usepackage{hyperref}

 \hypersetup{pdfstartview=FitH,  linkcolor=linkcolor,urlcolor=urlcolor, colorlinks=true}
 % Цвета для гиперссылок
\definecolor{linkcolor}{HTML}{799B03}
\definecolor{urlcolor}{HTML}{799B03} 

%\newfontfamily\subsubsectionfont[Color=Black]{Times New Roman}
\usepackage{amsthm,amsfonts,amsmath,amssymb,amscd,mathtools}  % Математические дополнения от AMS
\usepackage{indentfirst}                            % Красная строка
\usepackage[singlelinecheck=off,center]{caption}    % Многострочные подписи
\usepackage{soul}                                   % Поддержка переносоустойчивых подчёркиваний и зачёркиваний
\usepackage{icomma}                                 % Запятая в десятичных дробях
\usepackage{tocloft}
\usepackage{setspace}
\usepackage{fancyhdr}
%%% Цвета %%%

%% Номера формул
\mathtoolsset{showonlyrefs=true} % Показывать номера только у тех формул, на которые есть \eqref{} в тексте.

\setcounter{tocdepth}{2} % глубина оглавления

\usepackage{geometry} % Меняем поля страницы
\geometry{a4paper,top=2cm,bottom=2cm,left=2.5cm,right=1cm}
\linespread{1.0}                    % Одинарный интервал
\sloppy                             % Избавляемся от переполнений
\clubpenalty=10000                  % Запрещаем разрыв страницы после первой строки абзаца
\widowpenalty=10000                 % Запрещаем разрыв страницы после последней строки абзаца

\usepackage{multirow}
\usepackage{hhline}
\usepackage{enumitem}
\usepackage{marvosym}

\begin{document}
\begin{titlepage}
  \begin{center}

\begin{figure}
  \centering
    \includegraphics{shapkaKPI}
\end{figure}

 \Large \textbf{МІНІСТЕРСТВО ОСВІТИ І НАУКИ УКРАЇНИ}\\
      \textbf{ Національний технічний університет України\\
      <<Київський політехнічний інститут>>}\\

       \vspace{2 cm}
    
РОЗДІЛИ СУЧАСНОЇ КРИПТОЛОГІЇ\\
Комп’ютерний практикум №2\\
\vspace{1 cm}
Лінійний криптоаналіз блокових шифрів \\
    \vfill
    \vfill
  
\newlength{\ML}
\settowidth{\ML}{«\underline{\hspace{0.7cm}}» \underline{\hspace{2cm}}}
\hfill\begin{minipage}{0.4\textwidth}
 \textbf{Виконали:}\\
 студенти групи ФІ-73мп\\
 Грубіян Євгеній\\
 Свічкарьов Іван\\
 Варіант -- 4 \\
 \textbf{Прийняв:}\\
 Деркач А.Г\\
\end{minipage}%
\vfill
\vfill  
 

  
\begin{center}
  Київ\\
  2018
\end{center}
\end{center}
\end{titlepage}

\setcounter{page}{2} % начать нумерацию с номера 2

\section{Мета роботи}

Опанування сучасних методів криптоаналізу блокових шифрів, набуття навичок у
дослідженні стійкості блокових шифрів до лінійного криптоаналізу.

\section{Постановка задачі}

\begin{enumerate}

\item Реалізувати методом <<гілок та границь>> пошук п’ятираундових лінійних
апроксимацій шифру Хейса із великим потенціалом. 

\item Реалізувати атаку на перший раундовий ключ шифру Хейса. 

\end{enumerate}


\section{Хід роботи}

Пошук п’ятираундових лінійних апроксимацій шифру Хейса із великим потенціалом відбувався за методов <<гілок та границь>>. 
Для прискорення пошуку були обрані такі порогові значення для раундових списків потенціалів : $\{ 0.00015, 0.00015, 0.00015, 0.00015, 0.000015 \}$.\par
Початкові різниці $\alpha$ обиралися із однією ненульовою тетрадою.\par
Для успішної атаки треба було накопичити необхідну кількість $N$ пар <<відкритий текст-шифротекст>> $(X_0,X_{r})$ та -- $M$ апроксимацій. \par
Ключ однозначно знаходився при:
\begin{enumerate}

\item $N=8000, M=300$, а лічильник ключа
переходу через поріг $u = 0.7 \cdot \hat{u}_{max}(k)$, де
\[
\hat{u}(k) = | \# \{(x,y): \alpha \cdot X_1 \oplus \beta \cdot X_{r} =0 \} - \# \{(x,y): \alpha \cdot X_1 \oplus \beta X_{r} = 1 \} | 
\]
 був у 2 рази більше за другий у вихідному списку кандидатів у ключ. 

\item $N=4000, M=1000$, а лічильник переходу через поріг $u = 0.7 \cdot \hat{u}_{max}(k)$ був на $14\%$ більше за другий у вихідному списку кандидатів у ключ. 

\end{enumerate}

При $N=4000, M=300$ -- iстинний ключ був третім у списку кандидатів.


\section{Висновки}

\begin{itemize}

\item Від обраних параметрів $M,N,u$ залежить однозначне знаходження ключа.
\item Було знайдено ключ: $k_0$ = 0x3937 із використання $8000$ пар текстів та $300$ апроксимацій. 

\end{itemize}


\end{document}