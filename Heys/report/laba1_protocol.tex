\documentclass[12pt,a4paper]{article}
\usepackage[utf8]{inputenc}
\usepackage[english,russian,ukrainian]{babel}
\usepackage{indentfirst}
\usepackage{misccorr}
\usepackage{graphicx}
\usepackage{amsmath}
\usepackage{euscript}  
\usepackage{slashbox}
\renewcommand{\baselinestretch}{1}

\usepackage{xcolor}
\usepackage{hyperref}

 \hypersetup{pdfstartview=FitH,  linkcolor=linkcolor,urlcolor=urlcolor, colorlinks=true}
 % Цвета для гиперссылок
\definecolor{linkcolor}{HTML}{799B03}
\definecolor{urlcolor}{HTML}{799B03} 

%\newfontfamily\subsubsectionfont[Color=Black]{Times New Roman}
\usepackage{amsthm,amsfonts,amsmath,amssymb,amscd,mathtools}  % Математические дополнения от AMS
\usepackage{indentfirst}                            % Красная строка
\usepackage[singlelinecheck=off,center]{caption}    % Многострочные подписи
\usepackage{soul}                                   % Поддержка переносоустойчивых подчёркиваний и зачёркиваний
\usepackage{icomma}                                 % Запятая в десятичных дробях
\usepackage{tocloft}
\usepackage{setspace}
\usepackage{fancyhdr}
%%% Цвета %%%

%% Номера формул
\mathtoolsset{showonlyrefs=true} % Показывать номера только у тех формул, на которые есть \eqref{} в тексте.

\setcounter{tocdepth}{2} % глубина оглавления

\usepackage{geometry} % Меняем поля страницы
\geometry{a4paper,top=2cm,bottom=2cm,left=2.5cm,right=1cm}
\linespread{1.0}                    % Одинарный интервал
\sloppy                             % Избавляемся от переполнений
\clubpenalty=10000                  % Запрещаем разрыв страницы после первой строки абзаца
\widowpenalty=10000                 % Запрещаем разрыв страницы после последней строки абзаца

\usepackage{multirow}
\usepackage{hhline}
\usepackage{enumitem}
\usepackage{marvosym}

\begin{document}
\begin{titlepage}
  \begin{center}

\begin{figure}
  \centering
    \includegraphics{shapkaKPI}
\end{figure}

 \Large \textbf{МІНІСТЕРСТВО ОСВІТИ І НАУКИ УКРАЇНИ}\\
      \textbf{ Національний технічний університет України\\
      “Київський політехнічний інститут”}\\

       \vspace{2 cm}
    
РОЗДІЛИ СУЧАСНОЇ КРИПТОЛОГІЇ\\
Комп’ютерний практикум №1\\
\vspace{1 cm}
Диференціальний криптоаналіз блокових шифрів \\
    \vfill
    \vfill
  
\newlength{\ML}
\settowidth{\ML}{«\underline{\hspace{0.7cm}}» \underline{\hspace{2cm}}}
\hfill\begin{minipage}{0.4\textwidth}
 \textbf{Виконали:}\\
 студенти групи ФІ-33\\
 Грубіян Євгеній\\
 Свічкарьов Іван\\
 Варіант -- 4 \\
 \textbf{Прийняв:}\\
 Деркач .\\
\end{minipage}%
\vfill
\vfill  
 

  
\begin{center}
  Київ\\
  2018
\end{center}
\end{center}
\end{titlepage}

\setcounter{page}{2} % начать нумерацию с номера 2

\section{Мета роботи}

Опанування сучасних методів криптоаналізу блокових шифрів, набуття навичок у дослідженні стійкості блокових шифрів до диференціального криптоаналізу

\section{Постановка задачі}

\begin{enumerate}
\item 
Реалізувати пошук високоімовірних п'ятираундових диференціалів шифру Хейса методом <<гілок та границь>>. Для пошуку рекомендується використовувати початкові різниці $\alpha$ із однією ненульовою тетрадою (це дає змогу максимізувати імовірності на перших етапах пошуку). Якщо у вихідній різниці будуть наявні нульові тетради, це може ускладнити проведення атаки: окремі біти ключа можуть не відновитись через брак статистичної інформації.
\item
Реалізувати атаку на сьомий раундовий ключ шифру Хейса. Для побудови атаки використати знайдені на попередньому кроці диференціали із високою імовірністю.
\end{enumerate}


\section{Хід роботи}




\begin{table}[h!]
\centering
\begin{tabular}{|c|c|c|c|c|c|c|c|c|c|c|c|c|c|c|c|c|}
\hline
\backslashbox{$\alpha$}{$\beta$}  & 0  & 1 & 2 & 3 & 4 & 5 & 6 & 7 & 8 & 9 & A & B & C & D & E & F \\ \hline
0 & 16 & 0 & 0 & 0 & 0 & 0 & 0 & 0 & 0 & 0 & 0 & 0 & 0 & 0 & 0 & 0 \\ \hline
1 & 0  & 2 & 0 & 2 & 0 & 2 & 0 & 2 & 0 & 0 & 2 & 2 & 0 & 0 & 2 & 2 \\ \hline
2 & 0  & 0 & 2 & 0 & 0 & 0 & 0 & 2 & 0 & 0 & 2 & 0 & 2 & 2 & 2 & 4 \\ \hline
3 & 0  & 0 & 2 & 2 & 2 & 2 & 0 & 0 & 2 & 0 & 0 & 2 & 2 & 0 & 0 & 2 \\ \hline
4 & 0  & 2 & 2 & 2 & 2 & 0 & 0 & 0 & 2 & 2 & 0 & 2 & 0 & 0 & 2 & 0 \\ \hline
5 & 0  & 4 & 2 & 0 & 2 & 0 & 2 & 2 & 2 & 0 & 0 & 0 & 0 & 0 & 0 & 2 \\ \hline
6 & 0  & 0 & 0 & 4 & 2 & 0 & 0 & 2 & 0 & 0 & 0 & 0 & 4 & 2 & 2 & 0 \\ \hline
7 & 0  & 0 & 0 & 2 & 0 & 0 & 2 & 0 & 2 & 2 & 0 & 2 & 0 & 4 & 0 & 2 \\ \hline
8 & 0  & 0 & 0 & 0 & 0 & 2 & 0 & 2 & 2 & 2 & 2 & 2 & 2 & 0 & 2 & 0 \\ \hline
9 & 0  & 2 & 0 & 0 & 0 & 2 & 4 & 0 & 0 & 0 & 0 & 2 & 2 & 2 & 2 & 0 \\ \hline
A & 0  & 0 & 2 & 2 & 0 & 4 & 2 & 2 & 0 & 4 & 0 & 0 & 0 & 0 & 0 & 0 \\ \hline
B & 0  & 0 & 2 & 2 & 2 & 0 & 2 & 0 & 0 & 2 & 2 & 0 & 4 & 0 & 0 & 0 \\ \hline
C & 0  & 2 & 0 & 0 & 2 & 2 & 0 & 2 & 0 & 4 & 2 & 0 & 0 & 2 & 0 & 0 \\ \hline
D & 0  & 0 & 0 & 0 & 2 & 0 & 2 & 0 & 2 & 0 & 2 & 0 & 0 & 0 & 4 & 4 \\ \hline
E & 0  & 0 & 2 & 0 & 2 & 0 & 2 & 2 & 0 & 0 & 2 & 4 & 0 & 2 & 0 & 0 \\ \hline
F & 0  & 4 & 2 & 0 & 0 & 2 & 0 & 0 & 4 & 0 & 2 & 0 & 0 & 2 & 0 & 0 \\ \hline
\end{tabular}
\caption{Таблиця диференціальних імовірностей 4-ого S-блоку}
\label{tab:sBoxDiffTab}
\end{table}

Пошук високоімовірних диференціалів відбувався за методов <<гілок та границь>>. 
Для прискорення пошуку були обрані такі порогові значення для імовірностей раундових диференціалів : $\{ 0.124, 0.00195, 0.0003, 0.00005, 0.00005 \}$.\par
Початкові різниці $\alpha$ обиралися із однією ненульовою тетрадою, а вихідні різниці відбиралися із відсутніми нульовими тетрадами. Загалом, щоб атака була успішною з високою імовірністю, можна взяти 3 самі імовірні диференціали із зазначеними $\alpha$ у таблиці \ref{tab:diffs}, використовуючи 16000 текстів. Щоб атака була успішною для деякого окремого диференцілу треба взяти $\frac{16}{p}$ текстів, де $p$ -- імовірність відповідного диференціалу.
Формат таблиці \ref{tab:diffs} -- {імовірність|кількість співпадінь $\beta$ із обрахунковою різницею пари текстів, розшифрованих на один раунд|у скількі разів ключ домінує над іншими}. Другий, та третій параметр зазначені чисто для того, щоб подивитися, яка доля співпадінь може припадати на бажаний ключ.


\begin{table}[]
\centering
\begin{tabular}{|c|c|c|c|c|c|c|}
\hline
\backslashbox{$\alpha$}{$\beta$}       & 1111                        & 2222                                               & 4264                        & 4444                        & 8888                        \\ \hline
0003 &                               & 337|40|1,42          &                                                             &                               & 138 | 22 | 1,66          \\ \hline
0005 & \textbf{}                     & 294|24|1,71          & \textbf{}                                          &                               & 133 | 14 | 1,40          \\ \hline
0006 &                               & \textbf{681|38|1,46} &                                                             &                               & \textbf{286 | 36 | 2,25} \\ \hline
0009 &                               & \textbf{493|40|2,00} &                                                              &                               & 209|24|2,00          \\ \hline
000B &                               & \textbf{424|46|1,64} &                                                              &                               & 165|32|1,60          \\ \hline
000D & \textbf{}                     & \textbf{273|30|2,14} & \textbf{}                                          & \textbf{}                     & 103|16|1,60          \\ \hline
0600 & \textbf{454|26|1,30} &            &                                                              &                               &                               \\ \hline
6000 & \textbf{375|36|2,00} &                                & \textbf{342|16|1,14} & \textbf{711|22|1,83} &                               \\ \hline
9000 & \textbf{260|22|2,20} &                                                             &                               & 530|12|1,20          &                               \\ \hline
B000 & \textbf{234|28|2,33} &                                                              & 209|16|1,60          & 471|20|1,43          &                               \\ \hline
D000 &                      &                                                       & \textbf{307|34|2,42} &                               \\ \hline
\end{tabular}
\caption{Таблиця диференціальній імовірностей, $10^{-6}$}
\label{tab:diffs}
\end{table}

Гарні початкові різниці $\alpha = \{0006,6000\}$, які обираючи саму імовірну вихідну різницю $\beta$, дають за 16000 текстів бажаний сьомий раундовий ключ $k_7 = dea7$. 


\section{Висновки}

\end{document}